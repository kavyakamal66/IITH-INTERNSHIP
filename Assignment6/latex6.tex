\documentclass[journal,12pt,twocolumn]{IEEEtran}
%

\usepackage{setspace}
\usepackage{gensymb}
\singlespacing
\usepackage[cmex10]{amsmath}
\usepackage{siunitx}
\usepackage{amsthm}

\usepackage{mathrsfs}

\usepackage{txfonts}
\usepackage{stfloats}

\usepackage{steinmetz}
\usepackage{cite}
\usepackage{cases}
\usepackage{subfig}
\usepackage{longtable}
\usepackage{multirow}
\usepackage{enumitem}
\usepackage{mathtools}
\usepackage{tikz}
\usepackage{circuitikz}
\usepackage{verbatim}
\usepackage{tfrupee}
\usepackage[breaklinks=true]{hyperref}
\usepackage{tkz-euclide} % loads  TikZ and tkz-base
\usetikzlibrary{calc,math}
\usetikzlibrary{fadings}
\usepackage{listings}
    \usepackage{color}                                            %%
    \usepackage{array}                                            %%
    \usepackage{longtable}                                        %%
    \usepackage{calc}                                             %%
    \usepackage{multirow}                                         %%
    \usepackage{hhline}                                           %%
    \usepackage{ifthen}                                           %%
  %optionally (for landscape tables embedded in another document): %%
    \usepackage{lscape}     
\usepackage{multicol}
\usepackage{chngcntr}
\usepackage{float}
\DeclareMathOperator*{\Res}{Res}

\renewcommand\thesection{\arabic{section}}
\renewcommand\thesubsection{\thesection.\arabic{subsection}}
\renewcommand\thesubsubsection{\thesubsection.\arabic{subsubsection}}

\renewcommand\thesectiondis{\arabic{section}}
\renewcommand\thesubsectiondis{\thesectiondis.\arabic{subsection}}
\renewcommand\thesubsubsectiondis{\thesubsectiondis.\arabic{subsubsection}}

\hyphenation{op-tical net-works semi-conduc-tor}
\def\inputGnumericTable{}                                 %%

\lstset{
%language=C,
frame=single, 
breaklines=true,
columns=fullflexible
}
\begin{document}
%


\newtheorem{theorem}{Theorem}[section]
\newtheorem{problem}{Problem}
\newtheorem{proposition}{Proposition}[section]
\newtheorem{lemma}{Lemma}[section]
\newtheorem{corollary}[theorem]{Corollary}
\newtheorem{example}{Example}[section]
\newtheorem{definition}[problem]{Definition}
\newcommand{\BEQA}{\begin{eqnarray}}
\newcommand{\EEQA}{\end{eqnarray}}
\newcommand{\define}{\stackrel{\triangle}{=}}
\bibliographystyle{IEEEtran}
\providecommand{\mbf}{\mathbf}
\providecommand{\pr}[1]{\ensuremath{\Pr\left(#1\right)}}
\providecommand{\qfunc}[1]{\ensuremath{Q\left(#1\right)}}
\providecommand{\sbrak}[1]{\ensuremath{{}\left[#1\right]}}
\providecommand{\lsbrak}[1]{\ensuremath{{}\left[#1\right.}}
\providecommand{\rsbrak}[1]{\ensuremath{{}\left.#1\right]}}
\providecommand{\brak}[1]{\ensuremath{\left(#1\right)}}
\providecommand{\lbrak}[1]{\ensuremath{\left(#1\right.}}
\providecommand{\rbrak}[1]{\ensuremath{\left.#1\right)}}
\providecommand{\cbrak}[1]{\ensuremath{\left\{#1\right\}}}
\providecommand{\lcbrak}[1]{\ensuremath{\left\{#1\right.}}
\providecommand{\rcbrak}[1]{\ensuremath{\left.#1\right\}}}
\theoremstyle{remark}
\newtheorem{rem}{Remark}
\newcommand{\sgn}{\mathop{\mathrm{sgn}}}
\providecommand{\abs}[1]{\l\vert#1\r\vert}
\providecommand{\abs}[1]{\lvert#1\rvert} 
\providecommand{\res}[1]{\Res\displaylimits_{#1}} 
\providecommand{\norm}[1]{\l\lVert#1\r\rVert}
%\providecommand{\norm}[1]{\lVert#1\rVert}
\providecommand{\mtx}[1]{\mathbf{#1}}
\providecommand{\mean}[1]{E\l[ #1 \r]}
\providecommand{\fourier}{\overset{\mathcal{F}}{ \rightleftharpoons}}
%\providecommand{\hilbert}{\overset{\mathcal{H}}{ \rightleftharpoons}}
\providecommand{\system}{\overset{\mathcal{H}}{ \longleftrightarrow}}
	%\newcommand{\solution}[2]{\textbf{Solution:}{#1}}
\newcommand{\solution}{\noindent \textbf{Solution: }}
\newcommand{\cosec}{\,\text{cosec}\,}
\providecommand{\dec}[2]{\ensuremath{\overset{#1}{\underset{#2}{\gtrless}}}}
\newcommand{\myvec}[1]{\ensuremath{\begin{pmatrix}#1\end{pmatrix}}}
\newcommand{\mydet}[1]{\ensuremath{\begin{vmatrix}#1\end{vmatrix}}}
\numberwithin{equation}{subsection}
\makeatletter
\@addtoreset{figure}{problem}
\makeatother
\let\StandardTheFigure\thefigure
\let\vec\mathbf
\renewcommand{\thefigure}{\theproblem}
\def\putbox#1#2#3{\makebox[0in][l]{\makebox[#1][l]{}\raisebox{\baselineskip}[0in][0in]{\raisebox{#2}[0in][0in]{#3}}}}
     \def\rightbox#1{\makebox[0in][r]{#1}}
     \def\centbox#1{\makebox[0in]{#1}}
     \def\topbox#1{\raisebox{-\baselineskip}[0in][0in]{#1}}
     \def\midbox#1{\raisebox{-0.5\baselineskip}[0in][0in]{#1}}
\vspace{3cm}
\title{ASSIGNMENT-6}
\author{Kavya Kamal}
\maketitle
\newpage
\bigskip
\renewcommand{\thefigure}{\theenumi}
\renewcommand{\thetable}{\theenumi}
%
  Download all python codes from 
\begin{lstlisting}
https://github.com/kavyakamal66/IITH-INTERNSHIP/blob/main/Assignment6/code6.py
\end{lstlisting}
%
and latex-tikz codes from 
%
\begin{lstlisting}
https://github.com/kavyakamal66/IITH-INTERNSHIP/blob/main/Assignment6/latex6.tex
\end{lstlisting}
%
\section{OPTIMIZATION 2.4}
 Find two numbers whose sum is 24 and whose product is as large as possible.
\section{SOLUTION}
Let x,y be two numbers
Given
\begin{align}
  x+y=24\\
  \implies y=24-x
\end{align}
For their product to be maximum
\begin{align}
    f(x)=xy=x(24-x)=24x-x^2 \label{eq:1}
\end{align}
\begin{lemma}
A function f(x) is said to be concave if following inequality is true for $\lambda \in [0,1] :$  \label{lemma1}
\begin{align}
    \lambda f(x_1) + (1-\lambda)f(x_2) \leq f(\lambda x_1 + (1-\lambda)x_2)
\end{align}
\end{lemma}

Checking convexity of $f(x)$ :

\begin{align}
    &\lambda(24{x_1}-{x_1}^2) + 
    (1-\lambda)\brak{24 {x_2}- {x_2}^2} \\ &\leq
    24(\lambda x_1 + (1-\lambda)x_2)-\brak{\lambda x_1 + (1-\lambda)x_2}^2 
\end{align}

\begin{align}
    \lambda(\lambda - 1)({x_1} -{x_2})^2 &\leq 0\\
    \implies \lambda(\lambda - 1)&\leq 0
\end{align}is true .
$\implies$
The function is concave.
Using gradient ascent method we can find its maxima,
    \begin{align}
        x_{n+1} &= x_n + \alpha \nabla f(x_n) \\
        \implies x_{n+1} &= x_n + \alpha \brak{-2x_n+24}
    \end{align}
    
    Taking $x_0=2,\alpha=0.001$ and precision= \\ 0.00000001,values obtained using python are:
    \begin{align}
        \boxed{\text{Maxima} =143.9999999999752 \approx 144 }\\
        \boxed{\text{Maxima Point} =  11.999995019260913\approx 12}
    \end{align}
We can verify this by the derivative test.
Since $f(x)$ is a concave function it has a maxima.
\begin{align}
\frac{df(x)}{dx}=-2x+24
\end{align}
Critical point :
\begin{align}
 \frac{df(x)}{dx}&=0\\
 -2x+24&=0\\
 x&=12
\end{align} 
is a critical point.And since $f(x)$ is a concave function there will be a maxima at $x=12$.
And the maxima is
\begin{align}
f(12)&=144
\end{align}
\numberwithin{figure}{section}
\begin{figure}[!h]
    \centering
    \includegraphics [width=\columnwidth]{fig6.png}
    \caption{$f(x)=24x-x^2$}
    \label{p(x)}	
\end{figure}
\end{document}
